\documentclass[12pt,a4paper]{article}
\usepackage[left=2cm,top=2cm,right=2cm,bottom=2cm,head=.5cm,foot=.5cm]{geometry}

\usepackage[hidelinks]{hyperref}
\usepackage{graphicx}
\usepackage{longtable}
\usepackage[table]{xcolor}

\definecolor{lightgray}{gray}{0.9}

\let\oldlongtable\longtable
\let\endoldlongtable\endlongtable

\def\zza{\global\let\zz\zzb
\fullwidthcolor{lightgray}}%

\def\zzb{\global\let\zz\zza}

\def\fullwidthcolor#1{\color{#1}\leaders\vrule\hskip\textwidth\hskip-\textwidth\kern0pt}
\def\resetLTcolor{\global\let\zz\zza}

\LTleft0pt
\LTright0pt

\newcommand{\angstrom}{\mbox{\normalfont\AA}}

\begin{document}

\begin{center}

\LARGE{Properties codebook for} \hspace{-4mm}
{\begingroup
\setbox0=\hbox{\includegraphics[width=0.3\textwidth]{../src/images/splash.png}}
\parbox{\wd0}{\box0}\endgroup}
\LARGE{version 1.001}

\rule{\textwidth}{1pt}
\end{center}

\noindent The following properties were recorded for the 20 common amino acids, using 
the\href{http://www.chemcomp.com/MOE-Molecular_Operating_Environment.htm}{
\textbf{Molecular Operating Environment}} (MOE) software package (version 2012.10), 
developed by the Chemical Computing Group (Montreal, Canada).

\resetLTcolor
\begin{longtable}{@{\zz}|p{0.24\textwidth}|p{0.71\textwidth}|}
\hline
\rowcolor{yellow!50}\multicolumn{1}{|c|}{\textbf{Property}} & \multicolumn{1}{|c|}{\textbf{Description}}\\
\hline
\endhead

\texttt{a\_acc} & Number of hydrogen bond acceptor atoms (not counting acidic 
atoms but counting atoms that are both hydrogen bond donors and acceptors such 
as -OH).\\ \hline

\texttt{a\_acid} & Number of acidic atoms.\\ \hline

\texttt{a\_aro} & Number of aromatic atoms.\\ \hline

\texttt{a\_base} & Number of basic atoms.\\ \hline

\texttt{a\_count} & Number of atoms (including implicit hydrogens). This is 
calculated as the sum of ($1 + h_i$) over all non-trivial atoms $i$.\\ \hline

\texttt{a\_don} & Number of hydrogen bond donor atoms (not counting basic atoms 
but counting atoms that are both hydrogen bond donors and acceptors such as 
-OH).\\ \hline

\texttt{a\_heavy} & Number of heavy atoms \#\{$Z_i | Z_i > 1$\}.\\ \hline

\texttt{a\_hyd} & Number of hydrophobic atoms.\\ \hline

\texttt{a\_IC} & Atom information content (total). This is calculated to be 
\texttt{a\_ICM} times $n$.\\ \hline

\texttt{a\_ICM} & Atom information content (mean). This is the entropy of the 
element distribution in the molecule (including implicit hydrogens but not 
lone pair pseudo-atoms). Let $n_i$ be the number of occurrences of atomic number 
$i$ in the molecule. Let $pi = n_i / n$ where $n$ is the sum of the $n_i$. The 
value of \texttt{a\_ICM} is the negative of the sum over all $i$ of $p_i$ log 
$p_i$.\\ \hline

\texttt{a\_nB} & Number of boron atoms \#\{$Z_i | Z_i = 5$\}.\\ \hline

\texttt{a\_nBr} & Number of bromine atoms \#\{$Z_i | Z_i = 35$\}.\\ \hline

\texttt{a\_nC} & Number of carbon atoms \#\{$Z_i | Z_i = 6$\}.\\ \hline

\texttt{a\_nCl} & Number of chlorine atoms \#\{$Z_i | Z_i = 17$\}.\\ \hline

\texttt{a\_nF} & Number of fluorine atoms \#\{$Z_i | Z_i = 9$\}.\\ \hline

\texttt{a\_nH} & Number of hydrogen atoms (including implicit hydrogens). This 
is calculated as the sum of $h_i$ over all non-trivial atoms $i$ plus the 
number of non-trivial hydrogen atoms.\\ \hline

\texttt{a\_nI} & Number of iodine atoms \#\{$Z_i | Z_i = 53$\}.\\ \hline

\texttt{a\_nN} & Number of nitrogen atoms \#\{$Z_i | Z_i = 7$\}.\\ \hline

\texttt{a\_nO} & Number of oxygen atoms \#\{$Z_i | Z_i = 8$\}.\\ \hline

\texttt{a\_nP} & Number of phosphorus atoms \#\{$Z_i | Z_i = 15$\}.\\ \hline

\texttt{a\_nS} & Number of sulfur atoms \#\{$Z_i | Z_i = 16$\}.\\ \hline

\texttt{AM1\_dipole} & The dipole moment calculated using the AM1 Hamiltonian 
\cite{MOPAC1993}.\\ \hline

\texttt{AM1\_E} & The total SCF energy (kcal/mol) calculated using the AM1 
Hamiltonian \cite{MOPAC1993}.\\ \hline

\texttt{AM1\_Eele} & The electronic energy (kcal/mol) calculated using the AM1 
Hamiltonian \cite{MOPAC1993}.\\ \hline

\texttt{AM1\_HF} & The heat of formation (kcal/mol) calculated using the AM1 
Hamiltonian \cite{MOPAC1993}.\\ \hline

\texttt{AM1\_HOMO} & The energy (eV) of the Highest Occupied Molecular Orbital 
calculated using the AM1 Hamiltonian \cite{MOPAC1993}.\\ \hline

\texttt{AM1\_IP} & The ionization potential (kcal/mol) calculated using the AM1 
Hamiltonian \cite{MOPAC1993}.\\ \hline

\texttt{AM1\_LUMO} & The energy (eV) of the Lowest Unoccupied Molecular Orbital 
calculated using the AM1 Hamiltonian \cite{MOPAC1993}.\\ \hline

\texttt{apol} & Sum of the atomic polarizabilities (including implicit 
hydrogens) with polarizabilities taken from \cite{CRC1994}.\\ \hline

\texttt{ASA} & Water accessible surface area calculated using a radius of 1.4 
$\angstrom$ for the water molecule. A polyhedral representation is used for each 
atom in calculating the surface area.\\ \hline

\texttt{ASA\_H} & Water accessible surface area of all hydrophobic ($|q_i|<0.2$) 
atoms.\\ \hline

\texttt{ASA\_minus} & Water accessible surface area of all atoms with negative 
partial charge (strictly less than 0).\\ \hline

\texttt{ASA\_P} & Water accessible surface area of all polar ($|q_i|\geq0.2$) 
atoms.\\ \hline

\texttt{ASA\_plus} & Water accessible surface area of all atoms with positive 
partial charge (strictly greater than 0).\\ \hline

\texttt{b\_1rotN} & Number of rotatable single bonds. Conjugated single bonds 
are not included (e.g. ester and peptide bonds).\\ \hline

\texttt{b\_1rotR} & Number of rotatable single bonds: \texttt{b\_1rotN} divided 
by \texttt{b\_heavy}.\\ \hline

\texttt{b\_ar} & Number of aromatic bonds.\\ \hline

\texttt{b\_count} & Number of bonds (including implicit hydrogens). This is 
calculated as the sum of ($d_i$/2 + $h_i$) over all non-trivial atoms $i$.\\ 
\hline

\texttt{b\_double} & Number of double bonds. Aromatic bonds are not considered 
to be double bonds.\\ \hline

\texttt{b\_heavy} & Number of bonds between heavy atoms.\\ \hline

\texttt{b\_rotN} & Number of rotatable bonds. A bond is rotatable if it has 
order 1, is not in a ring, and has at least two heavy neighbors.\\ \hline

\texttt{b\_rotR} & Fraction of rotatable bonds: \texttt{b\_rotN} divided by 
\texttt{b\_heavy}.\\ \hline

\texttt{b\_single} & Number of single bonds (including implicit hydrogens). 
Aromatic bonds are not considered to be single bonds.\\ \hline

\texttt{b\_triple} & Number of triple bonds (including implicit hydrogens). 
Aromatic bonds are not considered to be triple bonds.\\ \hline

\texttt{balabanJ} & Balaban's connectivity topological index 
\cite{Balaban1982}.\\ \hline

\texttt{BCUT\_PEOE\_0} & The BCUT descriptors \cite{Pearlman1998} are 
calculated from the eigenvalues of a modified adjacency matrix. Each $ij$ entry 
of the adjacency matrix takes the value 1/sqrt($b_{ij}$) where $b_{ij}$ is the 
formal bond order between bonded atoms $i$ and $j$. The diagonal takes the 
value of the PEOE partial charges. The resulting eigenvalues are sorted and 
the smallest, 1/3-ile, 2/3-ile and largest eigenvalues are reported.\\ \hline

\texttt{BCUT\_PEOE\_1} & See above.\\ \hline

\texttt{BCUT\_PEOE\_2} & See above.\\ \hline

\texttt{BCUT\_PEOE\_3} & See above.\\ \hline

\texttt{BCUT\_SLOGP\_0} & The BCUT descriptors using atomic contribution to 
logP (using the Wildman and Crippen SlogP method) instead of partial 
charge.\\ \hline

\texttt{BCUT\_SLOGP\_1} & See above.\\ \hline

\texttt{BCUT\_SLOGP\_2} & See above.\\ \hline

\texttt{BCUT\_SLOGP\_3} & See above.\\ \hline

\texttt{BCUT\_SMR\_0} & The BCUT descriptors using atomic contribution to molar 
refractivity (using the Wildman and Crippen SlogP method) instead of partial 
charge.\\ \hline

\texttt{BCUT\_SMR\_1} & See above.\\ \hline

\texttt{BCUT\_SMR\_2} & See above.\\ \hline

\texttt{BCUT\_SMR\_3} & See above.\\ \hline

\texttt{bpol} & Sum of the absolute value of the difference between atomic 
polarizabilities of all bonded atoms in the molecule (including implicit 
hydrogens) with polarizabilities taken from \cite{CRC1994}.\\ \hline

\texttt{CASA\_minus} & Negative charge weighted surface area, \texttt{ASA\_minus} 
times max \{ $q_i < 0$ \} \cite{Stanton1990}.\\ \hline

\texttt{CASA\_plus} & Positive charge weighted surface area, \texttt{ASA\_plus} 
times max \{ $q_i > 0$ \} \cite{Stanton1990}.\\ \hline

\texttt{chi0} & Atomic connectivity index (order 0) from \cite{Hall1991} and 
\cite{Kier1977}. This is calculated as the sum of 1/sqrt($d_i$) over all heavy 
atoms $i$ with $d_i > 0$.\\ \hline

\texttt{chi0\_C} & Carbon connectivity index (order 0). This is 
calculated as the sum of 1/sqrt($v_i$) over all carbon atoms $i$ with 
$d_i > 0$.\\ \hline

\texttt{chi0v} & Atomic valence connectivity index (order 0) from \cite{Hall1991} and 
\cite{Kier1977}. This is calculated as the sum of 1/sqrt($v_i$) over all heavy 
atoms $i$ with $v_i > 0$.\\ \hline

\texttt{chi0v\_C} & Carbon valence connectivity index (order 0). This is 
calculated as the sum of 1/sqrt($v_i$) over all carbon atoms $i$ with 
$v_i > 0$.\\ \hline

\texttt{chi1} & Atomic connectivity index (order 1) from \cite{Hall1991} and 
\cite{Kier1977}. This is calculated as the sum of 1/sqrt($d_id_j$) over all 
bonds between heavy atoms $i$ and $j$ where $i < j$.\\ \hline 

\texttt{chi1\_C} & Carbon connectivity index (order 1). This is calculated as 
the sum of 1/sqrt($d_id_j$) over all bonds between carbon atoms $i$ and $j$ where 
$i < j$.\\ \hline 

\texttt{chi1v} & Atomic valence connectivity index (order 1) from 
\cite{Hall1991} and \cite{Kier1977}. This is calculated as the sum of 
1/sqrt($v_iv_j$) over all bonds between heavy atoms $i$ and $j$ where $i < j$.\\ 
\hline

\texttt{chi1v\_C} & Carbon valence connectivity index (order 1). This is 
calculated as the sum of 1/sqrt($v_iv_j$) over all bonds between carbon atoms 
$i$ and $j$ where $i < j$.\\ \hline

\texttt{chiral} & The number of chiral centers.\\ \hline

\texttt{chiral\_u} & The number of unconstrained chiral centers.\\ \hline

\texttt{DASA} & Absolute value of the difference between \texttt{ASA\_plus} 
and \texttt{ASA\_minus}.\\ \hline

\texttt{DCASA} & Absolute value of the difference between \texttt{CASA\_plus} 
and \texttt{CASA\_minus} \cite{Stanton1990}.\\ \hline

\texttt{dens} & Mass density: molecular weight divided by van der Waals volume 
as calculated in the \texttt{vol} descriptor.\\ \hline

\texttt{density} & Molecular mass density: \texttt{Weight} divided by 
\texttt{vdw\_vol} (amu/$\angstrom^3$).\\ \hline

\texttt{diameter} & Largest value in the distance matrix \cite{Petitjean1992}.
\\ \hline

\texttt{dipole} & Dipole moment calculated from the partial charges of the 
molecule.\\ \hline

\texttt{dipoleX} & The x component of the dipole moment (external coordinates).\\ 
\hline

\texttt{dipoleY} & The y component of the dipole moment (external coordinates).\\ 
\hline

\texttt{dipoleZ} & The z component of the dipole moment (external coordinates).\\ 
\hline

\texttt{E} & Value of the potential energy.\\ \hline

\texttt{E\_ang} & Angle bend potential energy.\\ \hline

\texttt{E\_ele} & Electrostatic component of the potential energy.\\ \hline

\texttt{E\_nb} & Value of the potential energy with all bonded terms disabled.\\ 
\hline

\texttt{E\_oop} & Out-of-plane potential energy.\\ \hline

\texttt{E\_sol} & Solvation energy.\\ \hline

\texttt{E\_stb} & Bond stretch-bend cross-term potential energy.\\ \hline

\texttt{E\_str} & Bond stretch potential energy.\\ \hline

\texttt{E\_strain} & Local strain energy: the current energy minus the value 
of the energy at a near local minimum. The current energy is calculated as for 
the \texttt{E} descriptor. The local minimum energy is the value of the 
\texttt{E} descriptor after first performing an energy minimization.\\ \hline

\texttt{E\_tor} & Torsion (proper and improper) potential energy.\\ \hline

\texttt{E\_vdw} & Van der Waals component of the potential energy.\\ \hline

\texttt{FASA\_H} & Fractional \texttt{ASA\_H} calculated as 
\texttt{ASA\_H}/\texttt{ASA}.\\ \hline

\texttt{FASA\_minus} & Fractional \texttt{ASA\_minus} calculated as 
\texttt{ASA\_minus}/\texttt{ASA}.\\ \hline

\texttt{FASA\_P} & Fractional \texttt{ASA\_P} calculated as 
\texttt{ASA\_P}/\texttt{ASA}.\\ \hline

\texttt{FASA\_plus} & Fractional \texttt{ASA\_plus} calculated as 
\texttt{ASA\_plus}/\texttt{ASA}.\\ \hline

\texttt{FCASA\_minus} & Fractional \texttt{CASA\_minus} calculated as 
\texttt{CASA\_minus}/\texttt{ASA}.\\ \hline

\texttt{FCASA\_plus} & Fractional \texttt{CASA\_plus} calculated as 
\texttt{CASA\_plus}/\texttt{ASA}.\\ \hline

\texttt{FCharge} & Total charge of the molecule (sum of formal charges).\\ \hline

\texttt{GCUT\_PEOE\_0} & The GCUT descriptors are calculated from the 
eigenvalues of a modified graph distance adjacency matrix. Each $ij$ entry of 
the adjacency matrix takes the value 1/sqrt($d_{ij}$) where $d_{ij}$ is the 
(modified) graph distance between atoms $i$ and $j$. The diagonal takes the 
value of the PEOE partial charges. The resulting eigenvalues are sorted and 
the smallest, 1/3-ile, 2/3-ile and largest eigenvalues are reported.\\ \hline

\texttt{GCUT\_PEOE\_1} & See above.\\ \hline

\texttt{GCUT\_PEOE\_2} & See above.\\ \hline

\texttt{GCUT\_PEOE\_3} & See above.\\ \hline

\texttt{GCUT\_SLOGP\_0} & The GCUT descriptors using atomic contribution to 
logP (using the Wildman and Crippen SlogP method) instead of partial charge.\\ 
\hline

\texttt{GCUT\_SLOGP\_1} & See above.\\ \hline

\texttt{GCUT\_SLOGP\_2} & See above.\\ \hline

\texttt{GCUT\_SLOGP\_3} & See above.\\ \hline

\texttt{GCUT\_SMR\_0} & The GCUT descriptors using atomic contribution to 
molar refractivity (using the Wildman and Crippen SMR method) instead of 
partial charge.\\ \hline

\texttt{GCUT\_SMR\_1} & See above.\\ \hline

\texttt{GCUT\_SMR\_2} & See above.\\ \hline

\texttt{GCUT\_SMR\_3} & See above.\\ \hline

\texttt{glob} & Globularity, or inverse condition number (smallest 
eigenvalue divided by the largest eigenvalue) of the covariance matrix 
of atomic coordinates. A value of 1 indicates a perfect sphere while a 
value of 0 indicates a two- or one-dimensional object.\\ \hline

\texttt{Kier1} & First kappa shape index: $(n-1)^2 / m^2$ \cite{Hall1991}.\\ 
\hline

\texttt{Kier2} & Second kappa shape index: $(n-1)^2 / m^2$ \cite{Hall1991}.\\ 
\hline

\texttt{Kier3} & Third kappa shape index: $(n-1)(n-3)^2 / p_3^2$ for odd $n$, 
and $(n-3)(n-2)^2 / p_3^2$ for even $n$ \cite{Hall1991}.\\ \hline

\texttt{KierA1} & First alpha modified shape index: $s(s-1)^2 / m^2$ where 
$s = n + a$ \cite{Hall1991}.\\ \hline

\texttt{KierA2} & Second alpha modified shape index: $s(s-1)^2 / m^2$ where 
$s = n + a$ \cite{Hall1991}.\\ \hline

\texttt{KierA3} & Third alpha modified shape index: $(s-1)(s-3)^2 / p_3^2$ 
for odd $n$, and $(s-3)(s-2)^2 / p_3^2$ for even n where $s = n + a$ 
\cite{Hall1991}.\\ \hline

\texttt{KierFlex} & Kier molecular flexibility index: 
(\texttt{KierA1})(\texttt{KierA2})/$n$ \cite{Hall1991}.\\ \hline

\texttt{lip\_acc} & The number of O and N atoms.\\ \hline

\texttt{lip\_don} & The number of OH and NH atoms.\\ \hline

\texttt{lip\_druglike} & One if and only if \texttt{lip\_violation} $< 2$ 
otherwise zero.\\ \hline

\texttt{lip\_violation} & The number of violations of Lipinski's Rule of Five 
\cite{Lipinski2012}.\\ \hline

\texttt{logP\_par\_o\_div\_w\_par\_} & Log of the octanol/water partition 
coefficient (including implicit hydrogens). This property is calculated from a 
linear atom type model \cite{LOGP1998} with $r^2$ = 0.931, RMSE=0.393 on 
1,827 molecules.\\ \hline

\texttt{logS} & Log of the aqueous solubility (mol/L). This property is 
calculated from an atom contribution linear atom type model \cite{Hou2004} 
with $r^2$ = 0.90, {\raise.17ex\hbox{$\scriptstyle\sim$}}1,200 molecules.\\ \hline

\texttt{MNDO\_dipole} & The dipole moment calculated using the MNDO Hamiltonian 
\cite{MOPAC1993}.\\ \hline

\texttt{MNDO\_E} & The total SCF energy (kcal/mol) calculated using the MNDO 
Hamiltonian \cite{MOPAC1993}.\\ \hline

\texttt{MNDO\_Eele} & The electronic energy (kcal/mol) calculated using the 
MNDO Hamiltonian \cite{MOPAC1993}.\\ \hline

\texttt{MNDO\_HF} & The heat of formation (kcal/mol) calculated using the MNDO 
Hamiltonian \cite{MOPAC1993}.\\ \hline

\texttt{MNDO\_HOMO} & The energy (eV) of the Highest Occupied Molecular 
Orbital calculated using the MNDO Hamiltonian \cite{MOPAC1993}.\\ \hline

\texttt{MNDO\_IP} & The ionization potential (kcal/mol) calculated using the 
MNDO Hamiltonian \cite{MOPAC1993}.\\ \hline

\texttt{MNDO\_LUMO} & The energy (eV) of the Lowest Unoccupied Molecular 
Orbital calculated using the MNDO Hamiltonian \cite{MOPAC1993}.\\ \hline

\texttt{mr} & Molecular refractivity (including implicit hydrogens). This 
property is calculated from an 11 descriptor linear model \cite{MREF1998} 
with $r^2$ = 0.997, RMSE = 0.168 on 1,947 small molecules.\\ \hline

\texttt{mutagenic} & Indicator of the presence of potentially toxic groups. 
A non-zero value indicates that the molecule contains a mutagenic group. 
The table of mutagenic groups is based on the Kazius set \cite{Kazius2005}.\\ 
\hline

\texttt{nmol} & The number of molecules (connected components).\\ \hline

\texttt{npr1} & Normalized PMI ratio \texttt{pmi1}/\texttt{pmi3}.\\ \hline

\texttt{npr2} & Normalized PMI ratio \texttt{pmi2}/\texttt{pmi3}.\\ \hline

\texttt{opr\_brigid} & The number of rigid bonds from \cite{Oprea2000}.\\ \hline

\texttt{opr\_leadlike} & One if and only if \texttt{opr\_violation} $< 2$ 
otherwise zero.\\ \hline

\texttt{opr\_nring} & The number of ring bonds from \cite{Oprea2000}.\\ \hline

\texttt{opr\_nrot} & The number of rotatable bonds from \cite{Oprea2000}.\\ 
\hline

\texttt{opr\_violation} & The number of violations of Oprea's lead-like test 
\cite{Oprea2000}.\\ \hline

\texttt{PEOE\_PC\_minus} & Total negative partial charge: the sum of the 
negative $q_i$.\\ \hline

\texttt{PEOE\_PC\_plus} & Total positive partial charge: the sum of the 
positive $q_i$.\\ \hline

\texttt{PEOE\_RPC\_minus} & Relative negative partial charge: the smallest 
negative $q_i$ divided by the sum of the negative $q_i$.\\ \hline

\texttt{PEOE\_RPC\_plus} & Relative positive partial charge: the largest 
positive $q_i$ divided by the sum of the positive $q_i$.\\ \hline

\texttt{PEOE\_VSA\_FHYD} & Fractional hydrophobic van der Waals surface 
area. This is the sum of the $v_i$ such that $|q_i|$ is less than or 
equal to 0.2 divided by the total surface area. The $v_i$ are calculated 
using a connection table approximation.\\ \hline

\texttt{PEOE\_VSA\_FNEG} & Fractional negative van der Waals surface area. 
This is the sum of the $v_i$ such that $q_i$ is negative divided by the total 
surface area. The $v_i$ are calculated using a connection table 
approximation.\\ \hline

\texttt{PEOE\_VSA\_FPNEG} & Fractional negative polar van der Waals surface 
area. This is the sum of the $v_i$ such that $q_i$ is less than -0.2 divided 
by the total surface area. The $v_i$ are calculated using a connection 
table approximation.\\ \hline

\texttt{PEOE\_VSA\_FPOL} & Fractional polar van der Waals surface area. 
This is the sum of the $v_i$ such that $|q_i|$ is greater than 0.2 
divided by the total surface area. The $v_i$ are calculated using a 
connection table approximation.\\ \hline

\texttt{PEOE\_VSA\_FPOS} & Fractional positive van der Waals surface 
area. This is the sum of the $v_i$ such that $q_i$ is non-negative 
divided by the total surface area. The $v_i$ are calculated using a 
connection table approximation.\\ \hline

\texttt{PEOE\_VSA\_FPPOS} & Fractional positive polar van der Waals 
surface area. This is the sum of the $v_i$ such that $q_i$ is 
greater than 0.2 divided by the total surface area. The $v_i$ are 
calculated using a connection table approximation.\\ \hline

\texttt{PEOE\_VSA\_HYD} & Total hydrophobic van der Waals surface area. 
This is the sum of the $v_i$ such that $|q_i|$ is less than or equal 
to 0.2. The $v_i$ are calculated using a connection table approximation.\\ 
\hline

\texttt{PEOE\_VSA\_minus0} & Sum of $v_i$ where $q_i$ is in the range 
[-0.05,0.00).\\ \hline

\texttt{PEOE\_VSA\_minus1} & Sum of $v_i$ where $q_i$ is in the range 
[-0.10,-0.05).\\ \hline

\texttt{PEOE\_VSA\_minus2} & Sum of $v_i$ where $q_i$ is in the range 
[-0.15,-0.10).\\ \hline

\texttt{PEOE\_VSA\_minus3} & Sum of $v_i$ where $q_i$ is in the range 
[-0.20,-0.15).\\ \hline

\texttt{PEOE\_VSA\_minus4} & Sum of $v_i$ where $q_i$ is in the range 
[-0.25,-0.20).\\ \hline

\texttt{PEOE\_VSA\_minus5} & Sum of $v_i$ where $q_i$ is in the range 
[-0.30,-0.25).\\ \hline

\texttt{PEOE\_VSA\_minus6} & Sum of $v_i$ where $q_i$ is less than -0.3.\\ 
\hline

\texttt{PEOE\_VSA\_NEG} & Total negative van der Waals surface area. 
This is the sum of the $v_i$ such that $q_i$ is negative. The $v_i$ 
are calculated using a connection table approximation.\\ \hline

\texttt{PEOE\_VSA\_plus0} & Sum of $v_i$ where $q_i$ is in the range 
[0.00,0.05).\\ \hline

\texttt{PEOE\_VSA\_plus1} & Sum of $v_i$ where $q_i$ is in the range 
[0.05,0.10).\\ \hline

\texttt{PEOE\_VSA\_plus2} & Sum of $v_i$ where $q_i$ is in the range 
[0.10,0.15).\\ \hline

\texttt{PEOE\_VSA\_plus3} & Sum of $v_i$ where $q_i$ is in the range 
[0.15,0.20).\\ \hline

\texttt{PEOE\_VSA\_plus4} & Sum of $v_i$ where $q_i$ is in the range 
[0.20,0.25).\\ \hline

\texttt{PEOE\_VSA\_plus5} & Sum of $v_i$ where $q_i$ is in the range 
[0.25,0.30).\\ \hline

\texttt{PEOE\_VSA\_plus6} & Sum of $v_i$ where $q_i$ is greater than 0.3.\\ 
\hline

\texttt{PEOE\_VSA\_PNEG} & Total negative polar van der Waals surface 
area. This is the sum of the $v_i$ such that $q_i$ is less than -0.2. 
The $v_i$ are calculated using a connection table approximation.\\ \hline

\texttt{PEOE\_VSA\_POL} & Total polar van der Waals surface area. This 
is the sum of the $v_i$ such that $|q_i|$ is greater than 0.2. The $v_i$ 
are calculated using a connection table approximation.\\ \hline

\texttt{PEOE\_VSA\_POS} & Total positive van der Waals surface area. 
This is the sum of the $v_i$ such that $q_i$ is non-negative. The 
$v_i$ are calculated using a connection table approximation.\\ \hline

\texttt{PEOE\_VSA\_PPOS} & Total positive polar van der Waals surface 
area. This is the sum of the $v_i$ such that $q_i$ is greater than 
0.2. The $v_i$ are calculated using a connection table approximation.\\ \hline

\texttt{petitjean} & Value of (\texttt{diameter} - 
\texttt{radius})/\texttt{diameter}.\\ \hline

\texttt{petitjeanSC} & Petitjean graph Shape Coefficient as defined in 
\cite{Petitjean1992}: (\texttt{diameter} - \texttt{radius})/\texttt{radius}.\\ 
\hline

\texttt{PM3\_dipole} & The dipole moment calculated using the PM3 Hamiltonian 
\cite{MOPAC1993}.\\ \hline

\texttt{PM3\_E} & The total SCF energy (kcal/mol) calculated using the PM3 
Hamiltonian \cite{MOPAC1993}.\\ \hline

\texttt{PM3\_Eele} & The electronic energy (kcal/mol) calculated using the PM3 
Hamiltonian \cite{MOPAC1993}.\\ \hline

\texttt{PM3\_HF} & The heat of formation (kcal/mol) calculated using the PM3 
Hamiltonian \cite{MOPAC1993}.\\ \hline

\texttt{PM3\_HOMO} & The energy (eV) of the Highest Occupied Molecular 
Orbital calculated using the PM3 Hamiltonian \cite{MOPAC1993}.\\ \hline

\texttt{PM3\_IP} & The ionization potential (kcal/mol) calculated using 
the PM3 Hamiltonian \cite{MOPAC1993}.\\ \hline

\texttt{PM3\_LUMO} & The energy (eV) of the Lowest Unoccupied Molecular 
Orbital calculated using the PM3 Hamiltonian \cite{MOPAC1993}.\\ \hline

\texttt{pmi} & Principal moment of inertia.\\ \hline

\texttt{pmi1} & First diagonal element of diagonalized moment of inertia 
tensor.\\ \hline

\texttt{pmi2} & Second diagonal element of diagonalized moment of inertia 
tensor.\\ \hline

\texttt{pmi3} & Third diagonal element of diagonalized moment of inertia 
tensor.\\ \hline

\texttt{pmiX} & x component of the principal moment of inertia 
(external coordinates).\\ \hline

\texttt{pmiY} & y component of the principal moment of inertia 
(external coordinates).\\ \hline

\texttt{pmiZ} & z component of the principal moment of inertia 
(external coordinates).\\ \hline

\texttt{Q\_PC\_minus} & Total negative partial charge: the sum of the 
negative $q_i$.\\ \hline

\texttt{Q\_PC\_plus} & Total positive partial charge: the sum of the 
positive $q_i$.\\ \hline

\texttt{Q\_RPC\_minus} & Relative negative partial charge: the smallest 
negative $q_i$ divided by the sum of the negative $q_i$.\\ \hline

\texttt{Q\_RPC\_plus} & Relative positive partial charge: the largest 
positive $q_i$ divided by the sum of the positive $q_i$.\\ \hline

\texttt{Q\_VSA\_FHYD} & Fractional hydrophobic van der Waals surface 
area. This is the sum of the $v_i$ such that $|q_i|$ is less than or 
equal to 0.2 divided by the total surface area. The $v_i$ are 
calculated using a connection table approximation.\\ \hline

\texttt{Q\_VSA\_FNEG} & Fractional negative van der Waals surface 
area. This is the sum of the $v_i$ such that $q_i$ is negative 
divided by the total surface area. The $v_i$ are calculated using 
a connection table approximation.\\ \hline

\texttt{Q\_VSA\_FPNEG} & Fractional negative polar van der Waals 
surface area. This is the sum of the $v_i$ such that $q_i$ is 
less than -0.2 divided by the total surface area. The $v_i$ are 
calculated using a connection table approximation.\\ \hline

\texttt{Q\_VSA\_FPOL} & Fractional polar van der Waals surface 
area. This is the sum of the $v_i$ such that $|q_i|$ is greater 
than 0.2 divided by the total surface area. The $v_i$ are 
calculated using a connection table approximation.\\ \hline

\texttt{Q\_VSA\_FPOS} & Fractional positive van der Waals surface 
area. This is the sum of the $v_i$ such that $q_i$ is non-negative 
divided by the total surface area. The $v_i$ are calculated using 
a connection table approximation.\\ \hline

\texttt{Q\_VSA\_FPPOS} & Fractional positive polar van der Waals 
surface area. This is the sum of the $v_i$ such that $q_i$ is 
greater than 0.2 divided by the total surface area. The $v_i$ 
are calculated using a connection table approximation.\\ \hline

\texttt{Q\_VSA\_HYD} & Total hydrophobic van der Waals surface 
area. This is the sum of the $v_i$ such that $|q_i|$ is less 
than or equal to 0.2. The $v_i$ are calculated using a connection 
table approximation.\\ \hline

\texttt{Q\_VSA\_NEG} & Total negative van der Waals surface area. 
This is the sum of the $v_i$ such that $q_i$ is negative. The 
$v_i$ are calculated using a connection table approximation.\\ \hline

\texttt{Q\_VSA\_PNEG} & Total negative polar van der Waals surface 
area. This is the sum of the $v_i$ such that $q_i$ is less than 
-0.2. The $v_i$ are calculated using a connection table approximation.\\ \hline

\texttt{Q\_VSA\_POL} & Total polar van der Waals surface area. This 
is the sum of the $v_i$ such that $|q_i|$ is greater than 0.2. The 
$v_i$ are calculated using a connection table approximation.\\ \hline

\texttt{Q\_VSA\_POS} & Total positive van der Waals surface area. 
This is the sum of the $v_i$ such that $q_i$ is non-negative. 
The $v_i$ are calculated using a connection table approximation.\\ \hline

\texttt{Q\_VSA\_PPOS} & Total positive polar van der Waals surface 
area. This is the sum of the $v_i$ such that $q_i$ is greater than 
0.2. The $v_i$ are calculated using a connection table approximation.\\ \hline

\texttt{radius} & If $r_i$ is the largest matrix entry in row $i$ of the 
distance matrix $D$, then the radius is defined as the smallest of the $r_i$
\cite{Petitjean1992}.\\ \hline

\texttt{reactive} & Indicator of the presence of reactive groups. A 
non-zero value indicates that the molecule contains a reactive group. 
The table of reactive groups is based on the \cite{Oprea2000} set and 
includes metals, phospho-, N/O/S-N/O/S single bonds, thiols, acyl 
halides, Michael Acceptors, azides, esters, etc.\\ \hline

\texttt{rgyr} & Radius of gyration.\\ \hline

\texttt{rings} & The number of rings.\\ \hline

\texttt{RPC\_minus} & Relative negative partial charge: the smallest 
negative $q_i$ divided by the sum of the negative $q_i$.\\ \hline

\texttt{RPC\_plus} & Relative positive partial charge: the largest 
positive $q_i$ divided by the sum of the positive $q_i$.\\ \hline

\texttt{rsynth} & A value in [0,1] indicating the synthetic 
reasonableness, or feasibility, of the chemical structure. A value 
of 0 means it is unlikely that the molecule can be synthesized while 
a value of 1 means that it is likely that the molecule can be 
synthesized. The value reflects the fraction of heavy atoms in the 
molecule that can be traced back to starting materials fragments 
resulting from retrosynthetic disconnection rules.\\ \hline

\texttt{SlogP} & Log of the octanol/water partition coefficient 
(including implicit hydrogens). This property is an atomic 
contribution model \cite{Wildman1999} that calculates logP from 
the given structure; i.e. the correct protonation state 
(washed structures). Results may vary from the 
\texttt{logP\_par\_o\_div\_w\_par\_} descriptor. The training set 
for \texttt{SlogP} was {\raise.17ex\hbox{$\scriptstyle\sim$}}7000 
structures.\\ \hline

\texttt{SlogP\_VSA0} & Sum of $v_i$ such that $L_i\leq-0.4$.\\ \hline

\texttt{SlogP\_VSA1} & Sum of $v_i$ such that $L_i$ is in (-0.4,-0.2].\\ \hline

\texttt{SlogP\_VSA2} & Sum of $v_i$ such that $L_i$ is in (-0.2,0].\\ \hline

\texttt{SlogP\_VSA3} & Sum of $v_i$ such that $L_i$ is in (0,0.1].\\ \hline

\texttt{SlogP\_VSA4} & Sum of $v_i$ such that $L_i$ is in (0.1,0.15].\\ \hline

\texttt{SlogP\_VSA5} & Sum of $v_i$ such that $L_i$ is in (0.15,0.20].\\ \hline

\texttt{SlogP\_VSA6} & Sum of $v_i$ such that $L_i$ is in (0.20,0.25].\\ \hline

\texttt{SlogP\_VSA7} & Sum of $v_i$ such that $L_i$ is in (0.25,0.30].\\ \hline

\texttt{SlogP\_VSA8} & Sum of $v_i$ such that $L_i$ is in (0.30,0.40].\\ \hline

\texttt{SlogP\_VSA9} & Sum of $v_i$ such that $L_i > 0.40$.\\ \hline

\texttt{SMR} & Molecular refractivity (including implicit hydrogens). This 
property is an atomic contribution model \cite{Wildman1999} that assumes 
the correct protonation state (washed structures). The model was trained 
on {\raise.17ex\hbox{$\scriptstyle\sim$}}7000 structures and results may 
vary from the \texttt{mr} descriptor.\\ \hline

\texttt{SMR\_VSA0} & Sum of $v_i$ such that $R_i$ is in [0,0.11].\\ \hline

\texttt{SMR\_VSA1} & Sum of $v_i$ such that $R_i$ is in (0.11,0.26].\\ \hline

\texttt{SMR\_VSA2} & Sum of $v_i$ such that $R_i$ is in (0.26,0.35].\\ \hline

\texttt{SMR\_VSA3} & Sum of $v_i$ such that $R_i$ is in (0.35,0.39].\\ \hline

\texttt{SMR\_VSA4} & Sum of $v_i$ such that $R_i$ is in (0.39,0.44].\\ \hline

\texttt{SMR\_VSA5} & Sum of $v_i$ such that $R_i$ is in (0.44,0.485].\\ \hline

\texttt{SMR\_VSA6} & Sum of $v_i$ such that $R_i$ is in (0.485,0.56].\\ \hline

\texttt{SMR\_VSA7} & Sum of $v_i$ such that $R_i > 0.56$.\\ \hline

\texttt{std\_dim1} & Standard dimension 1: the square root of the largest 
eigenvalue of the covariance matrix of the atomic coordinates. A standard 
dimension is equivalent to the standard deviation along a principal 
component axis.\\ \hline

\texttt{std\_dim2} & Standard dimension 2: the square root of the second 
largest eigenvalue of the covariance matrix of the atomic coordinates. 
A standard dimension is equivalent to the standard deviation along a 
principal component axis.\\ \hline

\texttt{std\_dim3} & Standard dimension 3: the square root of the third 
largest eigenvalue of the covariance matrix of the atomic coordinates. 
A standard dimension is equivalent to the standard deviation along a 
principal component axis.\\ \hline

\texttt{TPSA} & Polar surface area ($\angstrom^2$) calculated using 
group contributions to approximate the polar surface area from connection 
table information only. The parameterization is that of \cite{Ertl2000}.\\ \hline

\texttt{VAdjEq} & Vertex adjacency information (equality): 
$-(1-f)log_2(1-f) - flog_2f$ where $f = (n^2 - m)/n^2$, $n$ is the number 
of heavy atoms and $m$ is the number of heavy-heavy bonds. If $f$ is not 
in the open interval (0,1), then 0 is returned.\\ \hline

\texttt{VAdjMa} & Vertex adjacency information (magnitude): $1 + log_2m$ 
where $m$ is the number of heavy-heavy bonds. If $m$ is zero, then zero 
is returned.\\ \hline

\texttt{VDistEq} & If $m$ is the sum of the distance matrix entries then 
\texttt{VdistEq} is defined to be the sum of $log_2m - p_i log_2p_i/m$ 
where $p_i$ is the number of distance matrix entries equal to $i$.\\ \hline

\texttt{VDistMa} & If $m$ is the sum of the distance matrix entries then 
\texttt{VDistMa} is defined to be the sum of $log_2m - D_{ij} log_2D_{ij}/m$ 
over all $i$ and $j$.\\ \hline

\texttt{vdw\_area} & Area of van der Waals surface ($\angstrom^2$) calculated 
using a connection table approximation.\\ \hline

\texttt{vdw\_vol} & Van der Waals volume ($\angstrom^3$) calculated using a 
connection table approximation.\\ \hline

\texttt{vol} & Van der Waals volume calculated using a grid approximation 
(spacing 0.75 $\angstrom$).\\ \hline

\texttt{VSA} & Van der Waals surface area. A polyhedral representation is 
used for each atom in calculating the surface area.\\ \hline

\texttt{vsa\_acc} & Approximation to the sum of VDW surface areas 
($\angstrom^2$) of pure hydrogen bond acceptors (not counting acidic 
atoms and atoms that are both hydrogen bond donors and acceptors such as 
-OH).\\ \hline

\texttt{vsa\_acid} & Approximation to the sum of VDW surface areas of 
acidic atoms ($\angstrom^2$).\\ \hline

\texttt{vsa\_base} & Approximation to the sum of VDW surface areas of 
basic atoms ($\angstrom^2$).\\ \hline

\texttt{vsa\_don} & Approximation to the sum of VDW surface areas of 
pure hydrogen bond donors (not counting basic atoms and atoms that are 
both hydrogen bond donors and acceptors such as -OH) ($\angstrom^2$).\\ \hline

\texttt{vsa\_hyd} & Approximation to the sum of VDW surface areas of 
hydrophobic atoms ($\angstrom^2$).\\ \hline

\texttt{vsa\_other} & Approximation to the sum of VDW surface areas 
($\angstrom^2$) of atoms typed as "other".\\ \hline

\texttt{vsa\_pol} & Approximation to the sum of VDW surface areas 
($\angstrom^2$) of polar atoms (atoms that are both hydrogen bond 
donors and acceptors), such as -OH.\\ \hline

\texttt{vsurf\_A} & Amphiphilic moment.\\ \hline

\texttt{vsurf\_CP} & Critical packing parameter.\\ \hline

\texttt{vsurf\_CW1} & Capacity factor (1).\\ \hline

\texttt{vsurf\_CW2} & Capacity factor (2).\\ \hline

\texttt{vsurf\_CW3} & Capacity factor (3).\\ \hline

\texttt{vsurf\_CW4} & Capacity factor (4).\\ \hline

\texttt{vsurf\_CW5} & Capacity factor (5).\\ \hline

\texttt{vsurf\_CW6} & Capacity factor (6).\\ \hline

\texttt{vsurf\_CW7} & Capacity factor (7).\\ \hline

\texttt{vsurf\_CW8} & Capacity factor (8).\\ \hline

\texttt{vsurf\_D1} & Hydrophobic volume (1).\\ \hline

\texttt{vsurf\_D2} & Hydrophobic volume (2).\\ \hline

\texttt{vsurf\_D3} & Hydrophobic volume (3).\\ \hline

\texttt{vsurf\_D4} & Hydrophobic volume (4).\\ \hline

\texttt{vsurf\_D5} & Hydrophobic volume (5).\\ \hline

\texttt{vsurf\_D6} & Hydrophobic volume (6).\\ \hline

\texttt{vsurf\_D7} & Hydrophobic volume (7).\\ \hline

\texttt{vsurf\_D8} & Hydrophobic volume (8).\\ \hline

\texttt{vsurf\_DD12} & Contact distances of \texttt{vsurf\_DDmin} (12).\\ \hline

\texttt{vsurf\_DD13} & Contact distances of \texttt{vsurf\_DDmin} (13).\\ \hline

\texttt{vsurf\_DD23} & Contact distances of \texttt{vsurf\_DDmin} (23).\\ \hline

\texttt{vsurf\_DW12} & Contact distances of \texttt{vsurf\_EWmin} (12).\\ \hline

\texttt{vsurf\_DW13} & Contact distances of \texttt{vsurf\_EWmin} (13).\\ \hline

\texttt{vsurf\_DW23} & Contact distances of \texttt{vsurf\_EWmin} (23).\\ \hline

\texttt{vsurf\_EDmin1} & Lowest hydrophobic energy (1).\\ \hline

\texttt{vsurf\_EDmin2} & Lowest hydrophobic energy (2).\\ \hline

\texttt{vsurf\_EDmin3} & Lowest hydrophobic energy (3).\\ \hline

\texttt{vsurf\_EWmin1} & Lowest hydrophilic energy (1).\\ \hline

\texttt{vsurf\_EWmin2} & Lowest hydrophilic energy (2).\\ \hline

\texttt{vsurf\_EWmin3} & Lowest hydrophilic energy (3).\\ \hline

\texttt{vsurf\_G} & Surface globularity.\\ \hline

\texttt{vsurf\_HB1} & H-bond donor capacity (1).\\ \hline

\texttt{vsurf\_HB2} & H-bond donor capacity (2).\\ \hline

\texttt{vsurf\_HB3} & H-bond donor capacity (3).\\ \hline

\texttt{vsurf\_HB4} & H-bond donor capacity (4).\\ \hline

\texttt{vsurf\_HB5} & H-bond donor capacity (5).\\ \hline

\texttt{vsurf\_HB6} & H-bond donor capacity (6).\\ \hline

\texttt{vsurf\_HB7} & H-bond donor capacity (7).\\ \hline

\texttt{vsurf\_HB8} & H-bond donor capacity (8).\\ \hline

\texttt{vsurf\_HL1} & Hydrophilic-Lipophilic (1).\\ \hline

\texttt{vsurf\_HL2} & Hydrophilic-Lipophilic (2).\\ \hline

\texttt{vsurf\_ID1} & Hydrophobic integy moment (1).\\ \hline

\texttt{vsurf\_ID2} & Hydrophobic integy moment (2).\\ \hline

\texttt{vsurf\_ID3} & Hydrophobic integy moment (3).\\ \hline

\texttt{vsurf\_ID4} & Hydrophobic integy moment (4).\\ \hline

\texttt{vsurf\_ID5} & Hydrophobic integy moment (5).\\ \hline

\texttt{vsurf\_ID6} & Hydrophobic integy moment (6).\\ \hline

\texttt{vsurf\_ID7} & Hydrophobic integy moment (7).\\ \hline

\texttt{vsurf\_ID8} & Hydrophobic integy moment (8).\\ \hline

\texttt{vsurf\_IW1} & Hydrophilic integy moment (1).\\ \hline

\texttt{vsurf\_IW2} & Hydrophilic integy moment (2).\\ \hline

\texttt{vsurf\_IW3} & Hydrophilic integy moment (3).\\ \hline

\texttt{vsurf\_IW4} & Hydrophilic integy moment (4).\\ \hline

\texttt{vsurf\_IW5} & Hydrophilic integy moment (5).\\ \hline

\texttt{vsurf\_IW6} & Hydrophilic integy moment (6).\\ \hline

\texttt{vsurf\_IW7} & Hydrophilic integy moment (7).\\ \hline

\texttt{vsurf\_IW8} & Hydrophilic integy moment (8).\\ \hline

\texttt{vsurf\_R} & Surface rugosity.\\ \hline

\texttt{vsurf\_S} & Interaction field surface area.\\ \hline

\texttt{vsurf\_V} & Interaction field volume.\\ \hline

\texttt{vsurf\_W1} & Hydrophilic volume (1).\\ \hline

\texttt{vsurf\_W2} & Hydrophilic volume (2).\\ \hline

\texttt{vsurf\_W3} & Hydrophilic volume (3).\\ \hline

\texttt{vsurf\_W4} & Hydrophilic volume (4).\\ \hline

\texttt{vsurf\_W5} & Hydrophilic volume (5).\\ \hline

\texttt{vsurf\_W6} & Hydrophilic volume (6).\\ \hline

\texttt{vsurf\_W7} & Hydrophilic volume (7).\\ \hline

\texttt{vsurf\_W8} & Hydrophilic volume (8).\\ \hline

\texttt{vsurf\_Wp1} & Polar volume (1).\\ \hline

\texttt{vsurf\_Wp2} & Polar volume (2).\\ \hline

\texttt{vsurf\_Wp3} & Polar volume (3).\\ \hline

\texttt{vsurf\_Wp4} & Polar volume (4).\\ \hline

\texttt{vsurf\_Wp5} & Polar volume (5).\\ \hline

\texttt{vsurf\_Wp6} & Polar volume (6).\\ \hline

\texttt{vsurf\_Wp7} & Polar volume (7).\\ \hline

\texttt{vsurf\_Wp8} & Polar volume (8).\\ \hline

\texttt{Weight} & Molecular weight (including implicit hydrogens) in 
atomic mass units with atomic weights taken from \cite{CRC1994}.\\ \hline

\texttt{wienerPath} & Wiener path number: half the sum of all the 
distance matrix entries as defined in \cite{Balaban1979} and 
\cite{Wiener1947}.\\ \hline

\texttt{wienerPol} & Wiener polarity number: half the sum of all 
the distance matrix entries with a value of 3 as defined in 
\cite{Balaban1979}.\\ \hline

\texttt{zagreb} & Zagreb index: the sum of $d_i^2$ over all heavy 
atoms $i$.\\ \hline

\end{longtable}

\bibliographystyle{abstract}
\bibliography{references}

\end{document}
